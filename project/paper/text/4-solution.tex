\section{OS-Level Task-Based Mechanism for Lightweight Manycore Processors}
\label{sec:solution}

	% Overview
	Tarefas são um caso especial de corotinas que encapsulam uma subrotina que
	pode ser executada independemente de quem as criou.  Diferentemente das
	corotinas, que possuem sua própria pilha de execução e contexto, a definição de
	tarefas proposta é desaclopada da necessidade de ter uma thread dedicada por
	tarefa. Neste ponto, introduzimos uma thread especial, nomeada de Dispatcher,
	que é um executor genérico de tarefas.  Seguindo o modelo Produtor-Consumidor,
	o Dispatcher consome tarefas de uma fila global de tarefas, onde solicitantes
	inserem atomicamente tarefas a serem realizadas.

	Uma Tarefa uma estrutura padronizada que guarda informações
	semânticas e de controle. Conforme exposto na Tabela X, as variáveis
	semânticas armazenam a função a ser executada, seus arguments, e um slot para
	guardar o retorno, caso desejado. Além disso, variáveis de controle,
	referentes a semântica do Dispatcher, são compostas pelo estado da tarefa,
	uma lista de tarefas dependêntes, quantidade de tarefas ativas das quais
	a tarefa atual depende, e um controle de sincronização com o solicitante.
	A Figura 3 ilustra diferentes conjuntos de tarefas que compõem um gráfico
	de dependência. Essa dependência nos permite introduzir comportamentos mais
	sofisticados e assíncronos, onde uma tarefa só estará pronta para executar
	quando todas as tarefas pais tiverem concluído suas execuções.

	O Pesudocódigo A sumariza o comportamento de um Dispatcher. Bloqueado em um
	semáforo, o Dispatcher aguardará novas solicitações de tarefas surgirem.
	Ao consumir uma tarefa, o Dispatcher atualiza seu estado e entra no escopo
	da tarefa. Neste ponto é possível notar a reutilização das pilhas de um
	Dispatcher por inúmeras tarefas independentes, pois quando o mesmo retornar
	do escopo da tarefa, o mesmo estará em seu estado inicial e apto a iniciar
	outra tarefa.
	A tarefa pode ainda sinalizar entre três valores, o estado de conclusão ao
	Dispatcher. Especificamente:

	\begin{description}
		\item[TASK\_RET\_SUCCESS] O Dispatcher concluirá a tarefa, sinalizando
			sua conclusão as tarefas dependentes e liberando o solicitante.

		\item[TASK\_RET\_AGAIN] A tarefa retornou com um erro mas é recuperável
			e a tarefa será reescalonada.

		\item[TASK\_RET\_STOP] A tarefa retornou com um erro, mas apesar de ser
			recuperável, ela precisa aguardar outra operação ser concluída.

		\item[TASK\_RET\_ERROR] A tarefa retornou um erro e é irrecuperável,
			o Dispatcher sinalizará o erro ao solicitante e propagará o erro
			para todas as tarefas dependêntes.
	\end{description}

	Para a implementação de novos Dispatchers para aumentar o paralelismo
	e diminuir o tempo de espera para execução de uma tarefa, basta criar novas
	threads que executem a função proposta que ela já suporta por definição
	multiplos Dispatchers.

	% Problem address
	A proposta do mecanismo baseado em tarefas no nível do OS endereça os
	problemas descritos na seção anterior da seguinte forma:
	% Solutions
	\begin{description}

		\item[Memory Utilization] A definição de múltiplos fluxos de execução
			em tarefas isoladas, ou em um grafo de dependência, reduz
			a utilização de páginas de memória pra threads baseado na quantidade
			de Dispatchers existentes. Se o kernel e a aplicação conseguirem
			isolar comportamentos simples sem a necessidade da criação de uma thread
			dedicada, mais memória estará disponível para armazenar dados úteis.

		\item[Data Locality] Configurando um Dispatcher a ficar sempre em
			apenas um núcleo, o mesmo poderá executar tarefas que compartilham as
			mesma estruturas de dados explorando a localidade dos dados.

		\item[Core Utilization] Neste ponto entra a importância de se definir
			o Dispatcher em nível do OS para que seja possível movê-lo para o
			núcleo mestre e ele compartilhe o tempo de execução com a thread mestre.
			Em um extremo, poderíamos até definir a própria thread que realiza
			chamadas de sistema em um Dispatcher.

		\item[Asynchronous Operations] É possível modelar tarefas para
			introduzir essa noção de envio assíncrono, deixando a
			responsabilidade de enviar os dados manualmente para o Dispatcher
			enquanto as threads continuam sua execução normalmente.
			A mesma ideia se aplica para as syscalls, onde operações que não são
			críticas podem ser realizadas de forma assíncrona, liberando as
			threads solicitantes.

		\item[Periodic Operations] É possível modelar tarefas que possuam um
			período ao qual elas são desbloqueadas e executam. Desta forma,
			eliminamos a necessidade de ter uma thread dedicada que fica
			esperando alguma condição para relizar operações específicas.
			Por exemplo, a leitura de uma comunicação da NoC que criará uma
			nova tarefa para execução de um procedimento remoto.

	\end{description}


