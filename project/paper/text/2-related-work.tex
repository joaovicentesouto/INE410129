\section{Related Work}
\label{sec:related-work}

	% Intro
	Existem diversas soluções propostas que visam tirar proveito do alto nível
	paralelismo e baixo consumo energético dos LWs sem introduzir um overhead
	indesejado ao sistema.

	% Specific
	Soluções específicas, que modelam todo o ambiente de execução entorno de um
	tipo de um paradigma de programação buscam eliminar problemas de concorrência
	existentes em sistemas multi-processador e preemptivos.
	Por exemplo, ambientes de execução baseados em tarefas ou corotinas exploram
	o paralelismo de tarefas que trabalham de forma cooperativa.
	Esses ambientes oferecem suporte a flexibilidade e um controle de grão fino
	das tarefas. É possível também mapear pradões de programação paralela, como
	OPENMP diretamente em conjuntos de tarefas/corotinas nesses sistemas.
	A cooperação entre as unidades de execução é um importante ponto de
	benefício desses sistemas, onde a troca de contexto é leve e não ocorrem
	involuntáriamente, tornando a execução mais preditiva.
	Entretanto, esses runtimes limitam a classe de aplicações que podem ser
	portadas para LWs. 

	% Generic
	Outras soluções, exploram desenvolver padrões e mecanismos no nível do
	kernel modeladas especificamente para LWs sem perder a compatibilidade com
	outros sistemas. Entretanto, alguns serviços do SO podem a vir requerer uma
	quantidade um footprint de memória excessivo, forçando a aplicação cliente
	a trabalhar com uma quantidade limitada.
	%
	% Our propose.
	Nós acreditamos que muitos desses serviços internos do SO podem ser
	modelados sobre uma perspectiva de tarefas solicitadas, periódicas e/ou
	assíncronas. Está solução é especialmente vantajosa em microkernel assímetricos
	por poderem utilizar núcleos ociosos para executar tais tarefas.
