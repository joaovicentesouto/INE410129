\section{Related Work}
\label{sec:related-work}

	% Intro
	Several proposed solutions aim to take advantage of the high level of
	parallelism and low energy consumption of the \lws without introducing an
	unwanted overhead to the system.
	%
	% Specific
	Specific solutions, which model the entire execution environment around
	a type of programming paradigm, seek to eliminate existing competition problems
	in multiprocessing and preemptive systems. For instance,
	task-based~\cite{Zhou:coroutine} or coroutine execution~\cite{Cesarini:task}
	runtimes exploit the task parallelism that works cooperatively. These runtimes
	support flexibility and fine-grained job control.  It is also possible to map
	parallel programming patterns, such as OpenMP directly into task/coroutine
	sets. Cooperation between the execution units is an essential benefit of these
	systems, where the context change is light and does not occur involuntarily,
	making the execution more predictive. However, these runtimes limit the class
	of applications that can be ported to \lws.

	% Generic
	Other solutions explore to develop standards and mechanisms at the
	kernel-level~\cite{Penna:Microkernel} modeled specifically for \lws without
	losing compatibility with other systems. However, some \os services may come to
	require an excessive footprint of memory, forcing the client application to
	work with a limited amount of memory.
	%
	% Our propose.
	We believe that many of these internal \os services can be modeled from
	a perspective of requested, periodic, and/or asynchronous tasks. This solution
	fits especially in asymmetric microkernels using the idle time of the master
	core and the sequential form that the syscalls are handled.

