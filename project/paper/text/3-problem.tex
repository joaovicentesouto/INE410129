\section{Problem Definition}
\label{sec:problem}

	% Overview
	O gerenciamento de memória interna de um cluster afeta todos os níveis de
	abstração no contexto dos LWs. Por exemplos, OSs precisam ser pequenos
	e leves para deixar a maior quantidade de memória disponível pra aplicação.
	Por outro lado, aplicações extremamente paralelas e concorrentes têm
	a necessidade de gerenciar manualmente a coerência dos dados manipulados.

	Para amenizar os problemas de coerência de
	cache existente nessas arquiteturas, Penna ETAL propôs um microkernel
	assímetrico para concentrar o gerenciamento e manipulação das estruturas do
	OS em um único núcleo. A Figura X ilustra a execução de uma chamada de
	sistema dentro do Nanvix. Quando um núcleo escravo deseja realizar uma operação
	privilegiada, este envia uma solicitação ao núcleo mestre, que por sua vez,
	realiza e retorna o resultado ao solicitante. Os problemas explorados neste
	trabalho estão contidos no escopo deste nível de OS.

	Além das dificuldades impostas pelo sistema de memória, nossa proposta
	busca atacar algumas das desvantagens do microkernel e do sistema de
	comunicação. As descrições a seguir sumarizam os principais problemas que nossa
	solução se propõe endereçar:
	
	% Problems 
	\begin{description}

		\item[Core Utilization:] Pela necessidade de ter um núcleo reservado
			para a thread de sistema, o microkernel assimétrico perde poder
			de processamento ao deixar o núcleo mestre ocioso entre
			solicitações de syscalls.

		\item[Memory Utilization:] Para cada novo fluxo de execução (thread),
			o sistema de memória deve reservar dois espaços de memória, um
			para a pilha de execução do usuário e outra reservada para
			o kernel. Geralmente, cada pilha possúi o tamanho de uma página de
			memória.
	
		\item[Data Locality:] Por causa da falta de suporte em hardware para
			coerência de cache, a disputa por uma região de memória
			compartilhada é custosa. Para entrar em uma região crítica, as
			threads precisam garantir que os endereços na cache estejam
			invalidados, forçando o acesso ao banco de memória local. Ao sair
			de uma região crítica, as threads precisam forçar flush dos
			dados modificados para que os outros possam ver a atualização.
			Por isso, o isolamento dos dados em um único núcleo para explorar
			a localidade dos dados na cache é tão importante.

		\item[Asynchronous Operations:] Pela simplificidade e redução de
			energia, LWs podem não possuir uma DMA dedicada para executar
			comunicação assíncrona. Deste modo, é responsabilidade da thread
			realizar pooling dos dados na NoC manualmente. Todas as chamadas de
			sistemas dentro do microkernel também bloqueiam as threads, mesmo que
			a mesma pudesse realizar operações independentes enquanto a syscall,
			que não é critica, é realizada.

		\item[Periodic Operations:] Para permitir a uma cluster seja monitorado
			ou receba comandos externos, é necessário que uma thread exista para
			solicitando verificações e aguardando mensagens externa, aumentando
			a necessidade de memória do OS. Entretanto, a existência dessas
			operações são essenciais para desenvolver serviços mais complexos,
			e.g., gerenciamento e migração de processos, invalidação de memória
			compartilhada e distribuída, e execução de procedimentos remotos.

	\end{description}
