\section{Problem Definition}
\label{sec:problem}

	% Overview
	The internal memory management of a cluster influences all abstraction
	levels in the context of \lws. For instance, \oses need to be small and
	lightweight to make the most memory available to the application. On the
	other hand, concurrent and parallel applications need to manually manage
	the consistency of the manipulated data.

	To alleviate the cache coherence problems existing in these architectures,
	Penna ETAL proposed an asymmetric microkernel to concentrate the management
	and manipulation of the \os structures in a single core. Figure
	X illustrates the execution of a system call within Nanvix. When a slave
	core wants to perform a privileged operation, it sends a request to the
	master core, which in turn, performs and returns the result to the
	requester. The problems explored in this work are contained within the
	scope of this level of the Nanvix.

	In addition to the difficulties imposed by the memory system, our proposal
	seeks to address some of the disadvantages of the microkernel and the
	communication system. The following descriptions summarize the main
	problems that our solution aims to address:

	% Problems
	\begin{description}
		\item[Memory Utilization] For each new thread, the memory
			system must reserve two memory spaces, for user and kernel stacks.
			Generally, each stack is the size of a memory page. Therefore,
			rapidly increasing the amount of memory needed to create new
			threads in the system.

		\item[Data Locality] Because of the lack of hardware support for cache
			coherence, competition for a shared memory region is costly. For
			instance, a thread must invalidate their memory cache to access the
			local memory bank before entering a critical region. When leaving
			it, the thread must force the written of the modified data to keep
			the coherence.

		\item[Core Utilization] Due to the need to have a core reserved for the
			master thread, the asymmetric microkernel loses processing power
			when it leaves the master core idle between syscall requests.

		\item[Asynchronous Operations] For power consumption reduction, \lws may
			not feature a dedicated DMA to perform asynchronous communication.
			Thus, a specific thread must to manually send the data through the
			\noc. All system calls within the microkernel are also blocking,
			even though a thread can perform independent operations in the
			meantime.

		\item[Periodic Operations] To allow a cluster to be monitored or
			receive external commands, a specific thread must request
			communication checks constantly. The existence of these operations
			is essential to develop more complex services, \eg process
			management and migration, invalidation of shared and distributed
			memory, and execution of remote procedures.
	\end{description}

