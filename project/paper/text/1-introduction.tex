\section{Introduction}
\label{sec:introduction}

	% Context
	A classe de processadores Lightweight Manycore destaca-se pela seu alto
	nível de paralelismo com baíxo consumo energético~\cite{francesquini2015}. As
	aplicações projetadas para executar nesses processadores tem a sua disposição
	centenas de núcleos agrupados e distribuídos em um único chip. Para endereçar
	a escalabilidade do sistema e prover a eficiência energética, LWs exibem
	características arquiteturais próprias que os diferem dos demais Manycores.
	Especificamente, eles integram:
	\begin{enumerate*}[label=(\roman*)]
		\item thousands of low-power cores with \mimd capability~\cite{Rossi2017};

		\item distributed memory architecture and small local memories shared by
			tightly-coupled groups of cores (aka \textit{clusters})~\cite{Bohnenstiehl2017};

		\item reliable and fast \nocs for message-passing~\cite{Bohnenstiehl2017}; and

		\item heterogeneous processing capabilities~\cite{Davidson2018}.
	\end{enumerate*}
	Alguns exemplos de LWs são A, B e C.

\iffalse
	%
	Esses processadores introduziram junto de suas vantagens, um conjunto de
	desafios no projeto e desenvolvimento de aplicações de baíxo e alto nível
	inserido pelas suas características arquiteturais. Algumas desses desafios
	são:
	\begin{itemize}
		\item Modelo de Programação Híbrida: uso do modelo de memória
			compartilhada para explorar o paralelismo dentro dos clusters e
			aplicação do modelo de troca de mensagens para lidar com
			comunicação entre clusters através da NoC~\cite{kelly2013};

		\item Falta de suporte em hardware para coerência de cache: para
			reduzir o consumo de energia requirindo o controle explícito da
			memória~\cite{francesquini2015};.

		\item Sistem de memória restritivo: a presença de múltiplos espaços de
			endereçamento e pequenas memórias locais requer o Data Tiling
			e Prefetching seja manipulado em software~\cite{Castro2016};

		\item Configurações heterogeneas: requer a programação de componentes
			distintos, dificultando o deploy de aplicações no
			LWs~\cite{Barbalace2015}.
	\end{itemize}
\fi

	% Motivation
	Tais características introduzem diversos desafios no desenvolvimento de
	aplicações de baíxo e alto nível. Por exemplo, 
	\begin{enumerate*}[label=(\roman*)]
		\item programação híbrida entre os modelos de memória compartilhada
			e troca de mensagens devido a natureza dos LWs~\cite{kelly2013};

		\item falta de suporte em hardware para coerência de
			cache\cite{francesquini2015};

		\item sistema de memória distribuído e restritivo~\cite{Castro2016}; e 

		\item programação de componentes heterogêneos~\cite{Barbalace2015}.
	\end{enumerate*}
	Neste contexto, diversos trabalhos propõem soluções em diversos níveis de
	abstração para amenizar as dificuldades e prover ao usuário final
	alternativas para o desenvolvimento dessas arquiteturas.
	%
	Soluções em nível do kernel, como PENNA-NANVIX, endereçam uma gama de
	problemas de baíxo nível para prover uma visão padronizada e facilitar
	a programabilidade e portabilidade para LWs.
	%
	Por outro lado, soluções como XXX, buscam em explorar um tipo específico de
	abstração e modelaam um esqueleto de solução que pode ser adaptado pra
	família de problemas endereçado.

	% Problem Definition
	Dentre todas as soluções, uma das principais dificuldades é como lidar
	com a quantidade reduzida de memória local dentro de um cluster. Vale
	ressaltar que essa memória muitas vezes não é exclusiva para o usuário.
	A memória local precisa armazenar códigos e estruturas de  dados da
	aplicação e de suas dependências, e.g., OS e/ou bibliotecas.
	Algo que consome uma quantidade considerável de memória é o suporte
	a múltiplos fluxos de programação. Por exemplo, alocação de páginas de
	memória para armazenar o contexto e pilha de execução de uma rotina.
	%
	% Goals and Contributions
	Neste contexto, o presente trabalho propõe uma abstração de task em nível
	de kernel para definir uma unidade genérica de execução. Similar a uma
	piscina de threads, uma thread do sistema aguarda receber essas unidades
	para executar.
	Com este mecanismo, buscamos atingir os seguintes objetivos
	e contribuíções:
	\begin{itemize}
		\item Projetar um mecanismo de task no nível do kernel para uso interno
			do OS e para a aplicação cliente.

		\item Diminuir a alocação de memória do sistema de threads ao
			reutilizar o contexto e pilha de execução de uma única thread;

		\item Melhorar a utilização de núcleos ociosos;

		\item Explorar a localidade dos dados na cache de um único núcleo;

		\item Habilitar operações assíncrona e/ou periódicas, e.g.,
			envio assíncrono em LWs que não possuem DMA.

		\item Facilitar a modelagem de funcionalidades internas do kernel.
	\end{itemize}
	Este trabalho faz parte do desenvolvimento do Nanvix, um sistema operacional
	distribuído projetado para endereçar as características dos LWs. O escopo
	do trabalho se limita ao aperfeiçoamento do um microkernel assimétrico que
	gerencia os recursos locais de um cluster.

	% Work Organization
	The remainder of this work is organized as follows.
	In Section~\ref{sec:related-work}, we discuss related work.
	In Section~\ref{sec:problem} we cover the problem definition.
	In Section~\ref{sec:solution}, we present our proposal.
	In Section~\ref{sec:platform}, we detail the experimental environment.
	In Section~\ref{sec:results}, we discuss our experimental results.
	In Section~\ref{sec:related-work} we discuss related works.
	In Section~\ref{sec:conclusions}, we draw our conclusions.
