\documentclass[conference]{IEEEtran}
\IEEEoverridecommandlockouts

% The preceding line is only needed to identify funding in the first footnote. If that is unneeded, please comment it out.
\usepackage{cite}
\usepackage{amsmath,amssymb,amsfonts}
%\usepackage{algorithmic}
\usepackage[]{algorithm2e}
\usepackage{graphicx}
\usepackage{textcomp}
\usepackage{xcolor}
\usepackage{lipsum}

\usepackage[]{siunitx}

\bibliographystyle{IEEEtran}

% Enumerations
\usepackage[inline]{enumitem}
\renewcommand*\descriptionlabel[1]{\hspace\labelsep\itshape #1}

% Spacing
\usepackage{xspace}

% Tables
\usepackage{booktabs}
\usepackage{multirow}
\newcommand{\specialcell}[3][c]
{\begin{tabular}[#1]{@{}#2@{}}#3\end{tabular}}

% Acronyms
\usepackage[acronym,nowarn]{glossaries}
\glsdisablehyper
%==============================================================================
% Institutions 
%=============================================================================

% PUC Minas
\newacronym{pucminas}{PUC Minas}{Pontifícia Universidade Católica de Minas Gerais}
	\newcommand{\pucminas}{\gls{pucminas}\xspace}

% UFSC
\newacronym{ufsc}{UFSC}{Universidade Federal de Santa Catarina}
	\newcommand{\ufsc}{\gls{ufsc}\xspace}

% UGA
\newacronym{uga}{UGA}{Univerité Grenoble Alpes}
	\newcommand{\uga}{\gls{uga}\xspace}

% Grenoble INP
\newacronym{inpg}{Grenoble INP}{Institut National Polytechnique de Grenoble}
	\newcommand{\inpg}{\gls{inpg}\xspace}

%==============================================================================
% Computer Architecture
%=============================================================================

% Manycores
\newcommand{\scc}{Intel Single-Cloud Computer\xspace}
\newcommand{\xeonphi}{Intel Xeon Phi\xspace}
\newcommand{\tilegx}{Tilera TILE-Gx100\xspace}
\newcommand{\tilepro}{Tilera TILE64\xspace}
\newcommand{\mppa}{Kalray MPPA-256\xspace}
\newcommand{\pulp}{PULP\xspace}
\newcommand{\taihulight}{Sunway SW26010\xspace}
\newcommand{\epiphany}{Adapteva Epiphany\xspace}
\newcommand{\optimsoc}{OpTiMSoC\xspace}
\newcommand{\hero}{HERO\xspace}
\newcommand{\celerity}{Celerity\xspace}

% Architecture Families
\newcommand{\amd}{x86-64\xspace}
\newcommand{\intel}{x86\xspace}
\newcommand{\openrisc}{OpenRISC\xspace}
\newcommand{\bostan}{Bostan\xspace}
\newcommand{\riscv}{RISC-V\xspace}

% ISA
\newacronym{isa}{ISA}{Instruction Set Architecture}
	\newcommand{\isa}{\gls{isa}\xspace}

% FIFO
\newacronym{fifo}{FIFO}{First-In First-Out}
	\newcommand{\fifo}{\gls{fifo}\xspace}

% VLIW
\newacronym{vliw}{VLIW}{Very Long Instruction Word}
	\newcommand{\vliw}{\gls{vliw}\xspace}

% Taxonomy
\newacronym{mimd}{MIMD}{Multiple Instruction Multiple Data}
	\newcommand{\mimd}{\gls{mimd}\xspace}
\newacronym{simd}{SIMD}{Single Instruction Multiple Data}
	\newcommand{\simd}{\gls{simd}\xspace}
\newacronym{numa}{NUMA}{Non-Uniform Memory Access}
	\newcommand{\numa}{\gls{numa}\xspace}
\newacronym{norma}{NoRMA}{No Remote Memory Access}
	\newcommand{\norma}{\gls{norma}\xspace}
\newacronym{amp}{AMP}{Asymmetric Multi-Processing}
	\newcommand{\amp}{\gls{amp}\xspace}
\newacronym{smp}{SMP}{Symmetric Multi-Processing}
	\newcommand{\smp}{\gls{smp}\xspace}
\newacronym{gpu}{GPU}{Graphics Processing Unit}
	\newcommand{\gpu}{\gls{gpu}\xspace}
	\newcommand{\gpus}{\glspl{gpu}\xspace}
\newacronym{fpga}{FPGA}{Field Programmable Gate Array}
	\newcommand{\fpga}{\gls{fpga}\xspace}
	\newcommand{\fpgas}{\glspl{fpga}\xspace}

% Core
\newacronym{pe}{PE}{Processing Element}
	\newcommand{\pe}{\gls{pe}\xspace}
	\newcommand{\pes}{\glspl{pe}\xspace}
\newacronym{rm}{RM}{Resource Manager}
	\newcommand{\rman}{\gls{rm}\xspace}
	\newcommand{\rmans}{\glspl{rm}\xspace}

\newcommand{\iocluster}{I/O Cluster\xspace}
\newcommand{\ioclusters}{I/O Clusters\xspace}
\newcommand{\ccluster}{Compute Cluster\xspace}
\newcommand{\cclusters}{Compute Clusters\xspace}

\newcommand{\cluster}{\textit{cluster}\xspace}
\newcommand{\clusters}{\textit{clusters}\xspace}

\newcommand{\microkernel}{\textit{microkernel}\xspace}
\newcommand{\multikernel}{\textit{multikernel}\xspace}

\newcommand{\lw}{\textit{lightweight manycore}\xspace}
\newcommand{\lws}{\textit{lightweight manycores}\xspace}

% MMU
\newacronym{mmu}{MMU}{Memory Management Unit}
	\newcommand{\mmu}{\gls{mmu}\xspace}
	\newcommand{\mmus}{\glspl{mmu}\xspace}

% Cache
\newcommand{\dcache}{d-cache\xspace}
\newcommand{\icache}{i-cache\xspace}

\newacronym{raw}{RaW}{Read after Write}
	\newcommand{\raw}{\gls{raw}\xspace}

% TLB
\newacronym{tlb}{TLB}{Translation Lookaside Buffer}
	\newcommand{\tlb}{\gls{tlb}\xspace}
	\newcommand{\tlbs}{\glspl{tlb}\xspace}

% J-TLB
\newacronym{jtlb}{JTLB}{Join TLB}
	\newcommand{\jtlb}{\gls{jtlb}\xspace}
	\newcommand{\jtlbs}{\gls{jtlb}\xspace}

% L-TLB
\newacronym{ltlb}{LTLB}{Locked TLB}
	\newcommand{\ltlb}{\gls{ltlb}\xspace}
	\newcommand{\ltlbs}{\glspl{ltlb}\xspace}

% I-TLB
\newacronym{itlb}{ITLB}{Instruction TLB}
	\newcommand{\itlb}{\gls{itlb}\xspace}

% D-TLB
\newacronym{dtlb}{DTLB}{Data TLB}
	\newcommand{\dtlb}{\gls{dtlb}\xspace}

% SRAM
\newacronym{sram}{SRAM}{Static Random Access Memory}
	\newcommand{\sram}{SRAM\xspace}

% DRAM
\newacronym{dram}{DRAM}{Dynamic Random Access Memory}
	\newcommand{\dram}{DRAM\xspace}

% SPD
\newacronym[plural=SPMs,firstplural=Scratchpad Memories (SPMs)]{spm}{SPM}{Scratchpad Memory}
	\newcommand{\spm}{\gls{spd}\xspace}
	\newcommand{\spms}{\glspl{spm}\xspace}

% DMA
\newacronym{dma}{DMA}{Direct Memory Access}
	\newcommand{\dma}{\gls{dma}\xspace}

% RMA
\newacronym{rma}{RMA}{Remote Memory Access}
	\newcommand{\rma}{\gls{rma}\xspace}

% NoC
\newacronym[plural=NoCs,firstplural=Redes-em-Chip (NoCs)]{noc}{NoC}{Rede-em-Chip}
	\newcommand{\noc}{\gls{noc}\xspace}
	\newcommand{\nocs}{\glspl{noc}\xspace}

% C-NoC
\newacronym{cnoc}{C-NoC}{Control NoC}
    \newcommand{\cnoc}{\gls{cnoc}\xspace}

% D-NoC
\newacronym{dnoc}{D-NoC}{Data NoC}
	\newcommand{\dnoc}{\gls{dnoc}\xspace}

%==============================================================================
% Operating Systems
%=============================================================================

% OS
\newacronym[plural=OSes,firstplural=Sistemas Operacionais (OSes)]{os}{OS}{Sistema Operacional}
	\newcommand{\os}{\gls{os}\xspace}
	\newcommand{\oses}{\glspl{os}\xspace}

% POSIX
\newacronym{posix}{POSIX}{Portable Operating System Interface}
	\newcommand{\posix}{\gls{posix}\xspace}

% Kernels
\newcommand{\barrelfish}{Barrelfish\xspace}
\newcommand{\linux}{Linux\xspace}
\newcommand{\unix}{Unix\xspace}
\newcommand{\rtems}{RTEMS\xspace}
\newcommand{\bsd}{BSD\xspace}
\newcommand{\nodeos}{NodeOS\xspace}
\newcommand{\nanvix}{Nanvix\xspace}
\newcommand{\mossca}{MOSSCA\xspace}
\newcommand{\popcorn}{Popcorn Linux\xspace}
\newcommand{\helios}{Helios\xspace}
\newcommand{\tessellation}{Tessellation\xspace}
\newacronym{lfour}{L4}{L4 Microkernel}
	\newcommand{\lfour}{\gls{lfour}\xspace}

% Operating Systems
\newcommand{\gnu}{GNU/Linux\xspace}
\newacronym{fos}{FOS}{Factored Operating System}
	\newcommand{\fos}{\gls{fos}\xspace}
\newacronym{nos}{nOS}{Nano-Sized Operating System}
	\newcommand{\nos}{\gls{nos}\xspace}

% HAL
\newacronym{hal}{HAL}{Camada de Abstração de Hardware}
	\newcommand{\hal}{\gls{hal}\xspace}
	\newcommand{\hals}{\glspl{hal}\xspace}

% API
\newacronym{api}{API}{Application Programming Interface}
	\newcommand{\api}{\gls{api}\xspace}
	\newcommand{\apis}{\glspl{api}\xspace}

% IPC
\newacronym{ipc}{IPC}{Inter-Process Communication}
	\newcommand{\ipc}{\gls{ipc}\xspace}

% COM
\newacronym{com}{COM}{Communication}
	\newcommand{\com}{\gls{com}\xspace}

% RMem
\newacronym{rmem}{RMem}{Remote Memory}
	\newcommand{\rmem}{\gls{rmem}\xspace}

% QoS
\newacronym{qos}{QoS}{Quality of Service}
	\newcommand{\qos}{\gls{qos}\xspace}

% RPC
\newacronym{rpc}{RPC}{Remote Producere Call}
	\newcommand{\rpc}{\gls{rpc}\xspace}


%==============================================================================
% High Performance Computing
%=============================================================================

% HPC
\newacronym{hpc}{HPC}{High-Performance Computing}
	\newcommand{\hpc}{\gls{hpc}\xspace}

% OpenMP
\newcommand{\openmp}{OpenMP}\xspace
\newcommand{\javaconcurrency}{Java Concurrency\xspace}
\newcommand{\pthreads}{POSIX Threads}\xspace
\newcommand{\tbb}{TBB}\xspace
\newcommand{\cilk}{Cilk}\xspace

% PGAS
\newacronym{pgas}{PGAS}{Partitioned Global Address Space}
	\newcommand{\pgas}{\gls{pgas}\xspace}
\newacronym{mpi}{MPI}{Message Passing Interface}
	\newcommand{\mpi}{\gls{mpi}\xspace}

%==============================================================================
% Other
%=============================================================================

\newcommand{\ie}{i.e.\xspace}
\newcommand{\eg}{e.g.\xspace}
\newcommand{\etal}{\textit{et al.}\xspace}

\newacronym{iid}{i.i.d}{Independent and Identically Distributed}
	\newcommand{\iid}{\gls{iid}\xspace}

\newacronym{anova}{ANOVA}{Analysis of Variance}
	\newcommand{\anova}{\gls{anova}\xspace}

%==============================================================================
% Benchmarks 
%=============================================================================


\newcommand{\microbenchmark}{\si{\micro}benchmark\xspace}
\newcommand{\Microbenchmark}{\si{\micro}Benchmark\xspace}
\newcommand{\microbenchmarks}{\si{\micro}benchmarks\xspace}
\newcommand{\Microbenchmarks}{\si{\micro}Benchmarks\xspace}

% Local Kernel Call
\newacronym{lkcall}{L-Kcall}{Local Kernel Call}
	\newcommand{\lkcall}{\gls{lkcall}\xspace}

% Remote Kernel Call
\newacronym{rkcall}{R-Kcall}{Remote Kernel Call}
	\newcommand{\rkcall}{\gls{rkcall}\xspace}

% Upcall
\newcommand{\upcall}{Upcall\xspace}

% Fork-Join
\newcommand{\forkjoin}{ForkJoin\xspace}

% Producer-Consumer
\newcommand{\buffer}{Buffer\xspace}

% Gaussian Filter
\newacronym{knoise}{KNoise}{Kernel Noise}
	\newcommand{\knoise}{\gls{knoise}\xspace}

\newcommand{\sync}{\textit{sync}\xspace}
\newcommand{\mailbox}{\textit{mailbox}\xspace}
\newcommand{\portal}{\textit{portal}\xspace}



\makeglossaries

% Bookmarks.
\usepackage{xcolor}
\usepackage[bookmarks,bookmarksopen,bookmarksdepth=2,colorlinks=true]{hyperref}
\hypersetup{
	citecolor   = blue,
	linkcolor   = blue,
	urlcolor    = blue,
    pdftitle    = { On the performance and Isolation of Asymmetric
	                Microkernel Design for Lightweight Manycores },
    pdfauthor   = {Pedro Henrique Penna},
    pdfcreator  = {Pedro Henrique Penna},
    pdfproducer = {Pedro Henrique Penna},
    pdfsubject  = {Operating Systems},
}

\linespread{0.95}

% Images
\usepackage{graphicx}
\graphicspath{{img/}}
\DeclareGraphicsExtensions{.pdf,.jpeg,.png}

% Review
\usepackage{todonotes}
\usepackage{lipsum}

\hyphenation{Kal-ray}
\hyphenation{light-weight}
\hyphenation{as-sess-es}

\def\BibTeX{{\rm B\kern-.05em{\sc i\kern-.025em b}\kern-.08em
    T\kern-.1667em\lower.7ex\hbox{E}\kern-.125emX}}

\linespread{0.95}

\begin{document}

\title{
	OS-Level Task-Based Mechanism for Lightweight Manycore Processors
}

\author{%
	\IEEEauthorblockN{João Vicente Souto}
	\IEEEauthorblockA{%
		\textit{UFSC}\\
		Florianópolis, Brazil\\
		joao.vicente.souto@grad.ufsc.br
	}
	\and
	\IEEEauthorblockN{Pedro Henrique Penna}
	\IEEEauthorblockA{%
		\textit{UGA, PUC Minas}\\
		Grenbole, France\\
		pedro.penna@sga.pucminas.br
	}
	\and
	\IEEEauthorblockN{Márcio Castro}
	\IEEEauthorblockA{%
		\textit{UFSC}\\
		Florianópolis, Brazil\\
		marcio.castro@ufsc.br
	}
}

\maketitle

	
\begin{abstract}
	LWs destacam-se pelo alto nível de paralelismo e eficiência energética.
	Entretanto, suas características arquiteturais introduzem diversos desafios
	na programação desses processadores.
	%
	Um dos principais problemas é a limitada quantidade de memória local que as
	aplicações tem disponível para trabalhar. Esta limitação precisa ser
	considera pelas soluções de software para LWs.
	%
	Neste contexto, este trabalho propõe a definição de um mecanismo de tarefas
	dentro de um microkernel assimétrico destinado a LWs. Este mecanismo
	possibilita a definição de multiplos fluxos de execução com baixo custo ao
	sistema de memória através da implementação de uma quantidade limitada de
	threads especiais.
	%
	Os resultados mostraram ...
	%
\end{abstract}

\begin{IEEEkeywords}
	HAL, Distributed Operating System, Lightweight Manycore, Kalray MPPA-256
\end{IEEEkeywords}

	\section{Introduction}
\label{sec:introduction}

	% Context

	% Motivation

	% Problem Definition

	% Goals and Contributions

	% Goal and Contributions

	% Work Organization
	ABC.

	\section{Related Work}
\label{sec:related-work}

	\lipsum[2-4]

	\section{Problem Definition}
\label{sec:problem}

	ABC.

	\section{OS-Level Task-Based Mechanism for Lightweight Manycore Processors}
\label{sec:solution}

	% Overview
	A abstração de Tarefa proposta podem ser vista como um caso especial de
	corotinas que encapsulam uma subrotina e pode ser executada independemente
	de quem as criou. Entretanto, diferentemente das corotinas, que possuem sua
	própria pilha de execução e contexto, a definição de tarefas proposta
	é desaclopada da necessidade de ter uma thread dedicada por tarefa. Neste
	ponto, introduzimos uma thread especial, nomeada de Dispatcher, que é um
	executor genérico de tarefas.  Seguindo o modelo Produtor-Consumidor,
	o Dispatcher consome tarefas de uma fila global de tarefas, onde
	solicitantes inserem atomicamente tarefas a serem realizadas.

	Uma Tarefa é uma estrutura padronizada que guarda informações semânticas
	e de controle. De um lado, as variáveis semânticas armazenam a função a ser
	executada, seus argumentos, e um slot para guardar o retorno, caso
	desejado. Por outro lado, variáveis de controle, referentes a semântica do
	Dispatcher, são compostas pelo estado da tarefa, uma lista de tarefas
	dependêntes, quantidade de tarefas pais ativas, e um controle de
	sincronização com o solicitante.  Os grafos de dependência permitem
	introduzir comportamentos e gerenciamentos mais sofisticados assíncronos,
	onde uma tarefa só estará pronta para executar quando todas as tarefas pais
	tiverem concluído suas execuções.

	\begin{algorithm}[b]
		\While{!shutdown}{
			Waits for a task\;
			Consume from global task queue\;
			Execute task function\;
			\Switch{ret}{
				\Case{TASK\_RET\_SUCCESS}{
					Complete the task and schedule children\;
				}
				\Case{TASK\_RET\_AGAIN}{
					Reschedule the task\;
				}
				\Case{TASK\_RET\_STOP}{
					Insert the task into a waiting queue\;
				}
				\Case{TASK\_RET\_ERROR}{
					Propagate the error and release all tasks\;
				}
			}
		}
		\caption{How to write algorithms}
	\end{algorithm}

	O Pesudocódigo A sumariza o comportamento de um Dispatcher. Bloqueado em um
	semáforo, o Dispatcher aguardará novas solicitações de tarefas surgirem.
	Ao consumir uma tarefa, o Dispatcher atualiza seu estado e entra no escopo
	da tarefa. Neste ponto é possível notar a reutilização das pilhas de um
	Dispatcher por inúmeras tarefas independentes, pois quando o mesmo retornar
	do escopo da tarefa, o mesmo estará em seu estado inicial e apto a iniciar
	outra tarefa.  A tarefa pode ainda sinalizar entre três valores, o estado
	de conclusão ao Dispatcher. Especificamente:

	\begin{description}
		\item[TASK\_RET\_SUCCESS] O Dispatcher concluirá a tarefa, sinalizando
			sua conclusão as tarefas dependentes e liberando o solicitante.

		\item[TASK\_RET\_AGAIN] A tarefa retornou com um erro mas é recuperável
			e a tarefa será reescalonada.

		\item[TASK\_RET\_STOP] A tarefa retornou com um erro, mas apesar de ser
			recuperável, ela precisa aguardar outra operação ser concluída.

		\item[TASK\_RET\_ERROR] A tarefa retornou um erro e é irrecuperável,
			o Dispatcher sinalizará o erro ao solicitante e propagará o erro
			para todas as tarefas dependêntes.
	\end{description}

	Para a implementação de novos Dispatchers para aumentar o paralelismo
	e diminuir o tempo de espera para execução de uma tarefa, basta criar novas
	threads que executem a função proposta que ela já suporta por definição
	multiplos Dispatchers.

	% Problem address
	A proposta do mecanismo baseado em tarefas no nível do OS endereça os
	problemas descritos na seção anterior da seguinte forma:
	% Solutions
	\begin{description}

		\item[Memory Utilization] A definição de múltiplos fluxos de execução
			em tarefas isoladas, ou em um grafo de dependência, reduz
			a utilização de páginas de memória pra threads baseado na
			quantidade de Dispatchers existentes. Se o kernel e a aplicação
			conseguirem isolar comportamentos simples sem a necessidade da
			criação de uma thread dedicada, mais memória estará disponível para
			armazenar dados úteis.

		\item[Data Locality] Configurando um Dispatcher a ficar sempre em
			apenas um núcleo, o mesmo poderá executar tarefas que compartilham
			as mesma estruturas de dados explorando a localidade dos dados.

		\item[Core Utilization] Neste ponto entra a importância de se definir
			o Dispatcher em nível do OS para que seja possível movê-lo para
			o núcleo mestre e ele compartilhe o tempo de execução com a thread
			mestre.  Em um extremo, poderíamos até definir a própria thread que
			realiza chamadas de sistema em um Dispatcher.

		\item[Asynchronous Operations] É possível modelar tarefas para
			introduzir essa noção de envio assíncrono, deixando
			a responsabilidade de enviar os dados manualmente para o Dispatcher
			enquanto as threads continuam sua execução normalmente.  A mesma
			ideia se aplica para as syscalls, onde operações que não são
			críticas podem ser realizadas de forma assíncrona, liberando as
			threads solicitantes.

		\item[Periodic Operations] É possível modelar tarefas que possuam um
			período ao qual elas são desbloqueadas e executam. Desta forma,
			eliminamos a necessidade de ter uma thread dedicada que fica
			esperando alguma condição para relizar operações específicas.  Por
			exemplo, a leitura de uma comunicação da NoC que criará uma nova
			tarefa para execução de um procedimento remoto.

	\end{description}

	A Figura A exemplifica interações dentro do Nanvix após a implementação
	do mecanismo de tarefas. Tarefas pediódicas e criadas pelo próprio kernel
	permitiram uma interação mais rica intra e inter-clusters. Como a thread
	mestre responsável por atender uma solicitação de cada vez, mover tal
	responsabilidade pro Dispatcher não introduzirá um gargalo no sistema além
	do que já existe. Por fim, tarefas solicitadas por usuários comuns podem
	e devem se intercalar entre si, o que nos leva a necessidade de um
	gerenciamento mais fino do que a ordem de inserção numa fila.


	\section{Experimental Environment}
\label{sec:platform}

	ABC.

	\section{Results}
\label{sec:results}

	% Methodology
	% Experiments and Parameters

	\begin{table}[b]
	\centering
	\caption{Benchmark parameters for experiments.}
	\label{tab:benchmarks-parameters}
	\begin{tabular}{lll}
	\toprule
			\textbf{Benchmark}   & \textbf{\#Tasks} & \textbf{\#Threads} \\
			\midrule
			Single Dispatcher    & 1 to 29          & 1                  \\
			Multiple Dispatchers & 1 to 29          & 14                 \\
			Threads              & 1 to 29          & 1 to 29            \\
			\bottomrule
	\end{tabular}
	\end{table}

	% Time
	% Memory

	\section{Conclusions}
\label{sec:conclusions}

	ABC.


	\bibliography{aux/references.bib}

\end{document}
