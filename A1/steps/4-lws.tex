\section{Architectures}

	\begin{frame}[fragile, t]{Lightweight Manycores Processors}
			\only<1>{
				\addfig[\small{Overview of a Manycore}][width=.75\linewidth]{lightweight-manycore.pdf}

				\vspace{-0.25cm}

				\begin{itemize}
					\item \textbf{Hundreds of Lightweight Cores}
					\begin{itemize}
						\item Expose Massive thread-level parallelism
						\item Feature low-power consumption
						\item Target MIMD workloads
					\end{itemize}
					\item Distributed Memory Architecture
					\item On-Chip Heterogeneity
				\end{itemize}
			}
			\only<2>{
				\addfig[\small{Overview of a Manycore}][width=.75\linewidth]{lightweight-manycore.pdf}

				\vspace{-0.25cm}

				\begin{itemize}
					\item Hundreds of Lightweight Cores
					\item \textbf{Distributed Memory Architecture}
					\begin{itemize}
						\item Grants scalability
						\item Relies on a Network-on-Chip (NoC)
						\item Has constrained memory systems
					\end{itemize}
					\item On-Chip Heterogeneity
				\end{itemize}
			}
			\only<3>{
				\addfig[\small{Overview of a Manycore}][width=.75\linewidth]{lightweight-manycore.pdf}

				\vspace{-0.25cm}

				\begin{itemize}
					\item Hundreds of Lightweight Cores
					\item Distributed Memory Architecture
					\item \textbf{On-Chip Heterogeneity}
					\begin{itemize}
						\item Features different components
					\end{itemize}
				\end{itemize}
			}

		% Lightweight Manycores diferem dos multicores e manycores tradicionais em vários pontos.
		% Eles integram centenas de núcleos de baixa potência agrupados em clusters;
		% Trabalham com cargas de trabalho com múltiplos fluxos de instrução e dados;
		% Dependem de uma rede-em-chip para comunicação entre cluster, o que força o uso de um modelo de programação híbrido, ou seja, memória compartilhada e troca de mensagens;
		% Apresentam sistemas de memória restritívos, com pequenos e multiplos espaços de endereçamento sem coerência de cache;
		% E possuem geralmente componentes heterogêneos.

	\end{frame}

	\subsection{Kalray MPPA-256}

		\begin{frame}[fragile]{Kalray MPPA-256}{A Lightweight Manycore Processor}


			\begin{columns}[totalwidth=\linewidth]
				\column{0.55\linewidth}

					% \begin{itemize}
					% 	\item \textbf{High-Performance and Low-Power Consunption}
					% \end{itemize}

					\begin{itemize}
						\item \textbf{288 processing cores}
						\begin{itemize}
							\item {16 Compute Cluster (CC)}
							\item {4 I/O Cluster (IO)}
						\end{itemize}
					\end{itemize}

					\vspace{0.5cm}

					\begin{itemize}
						\item \textbf{Data NoC (D-NoC)}
						\begin{itemize}
							\item 256 RX slots
							\item 8 TX channels
							\item 8 $\mu$threads for async TX
						\end{itemize}
					\end{itemize}

					\vspace{0.5cm}

					\begin{itemize}
						\item \textbf{Control NoC (C-NoC)}
						\begin{itemize}
							\item 128 RX slots
							\item 4 TX channels
						\end{itemize}
					\end{itemize}

				\column{0.45\linewidth}
					\addfig[][width=.63\linewidth]{arch-mppa-2.pdf}
			\end{columns}

			% O processador MPPA-256 é um Lightweight Manycore de alto desempenho e baixo consumo energético.
			% Ele possui 288 núcleos de propósito geral agrupados em 16 clusters de computação e 4 clusters de entrada e saída.
		\end{frame}

		% \begin{frame}[fragile]{Kalray MPPA-256 Communication Resources}

		% 	% A comunicação é realizada através de duas NoCs distintas, uma para dados e outra para comandos.
		% 	% Cada NoC apresenta um conjunto de recursos para envio ou recebimento.
		% 	% Entretanto, a CNoC permite apenas a troca de mensagens de 64 bits.
		% \end{frame}


% LocalWords:  template cls standalone GitHub Overleaf bugfixes SVGs
% LocalWords:  Re-empacotamento fontsize Makefile pdflatex imgs PDFs
% LocalWords:  shell-escape frames SVG brazil english lapesd-slides
% LocalWords:  disabletodonotes todonotes TODO's backup showbackup
% LocalWords:  hidebackup abntexcite abntex natbib nobib titleframe
% LocalWords:  frame showsections sidebar stopcountingframes default
% LocalWords:  thanksframe Thank You Questions referencesframe titulo
% LocalWords:  bibfiles pholder todonote placeholder inline addfig
% LocalWords:  opts graphicx addfiglw width Citations dijkstra Direct
% LocalWords:  Closure Parallel dynamic scheduling DoImportantStuff
% LocalWords:  lccp merged cell svg pdf
